
% Default to the notebook output style

    


% Inherit from the specified cell style.




    
\documentclass[11pt]{article}

    
    
    \usepackage[T1]{fontenc}
    % Nicer default font (+ math font) than Computer Modern for most use cases
    \usepackage{mathpazo}

    % Basic figure setup, for now with no caption control since it's done
    % automatically by Pandoc (which extracts ![](path) syntax from Markdown).
    \usepackage{graphicx}
    % We will generate all images so they have a width \maxwidth. This means
    % that they will get their normal width if they fit onto the page, but
    % are scaled down if they would overflow the margins.
    \makeatletter
    \def\maxwidth{\ifdim\Gin@nat@width>\linewidth\linewidth
    \else\Gin@nat@width\fi}
    \makeatother
    \let\Oldincludegraphics\includegraphics
    % Set max figure width to be 80% of text width, for now hardcoded.
    \renewcommand{\includegraphics}[1]{\Oldincludegraphics[width=.8\maxwidth]{#1}}
    % Ensure that by default, figures have no caption (until we provide a
    % proper Figure object with a Caption API and a way to capture that
    % in the conversion process - todo).
    \usepackage{caption}
    \DeclareCaptionLabelFormat{nolabel}{}
    \captionsetup{labelformat=nolabel}

    \usepackage{adjustbox} % Used to constrain images to a maximum size 
    \usepackage{xcolor} % Allow colors to be defined
    \usepackage{enumerate} % Needed for markdown enumerations to work
    \usepackage{geometry} % Used to adjust the document margins
    \usepackage{amsmath} % Equations
    \usepackage{amssymb} % Equations
    \usepackage{textcomp} % defines textquotesingle
    % Hack from http://tex.stackexchange.com/a/47451/13684:
    \AtBeginDocument{%
        \def\PYZsq{\textquotesingle}% Upright quotes in Pygmentized code
    }
    \usepackage{upquote} % Upright quotes for verbatim code
    \usepackage{eurosym} % defines \euro
    \usepackage[mathletters]{ucs} % Extended unicode (utf-8) support
    \usepackage[utf8x]{inputenc} % Allow utf-8 characters in the tex document
    \usepackage{fancyvrb} % verbatim replacement that allows latex
    \usepackage{grffile} % extends the file name processing of package graphics 
                         % to support a larger range 
    % The hyperref package gives us a pdf with properly built
    % internal navigation ('pdf bookmarks' for the table of contents,
    % internal cross-reference links, web links for URLs, etc.)
    \usepackage{hyperref}
    \usepackage{longtable} % longtable support required by pandoc >1.10
    \usepackage{booktabs}  % table support for pandoc > 1.12.2
    \usepackage[inline]{enumitem} % IRkernel/repr support (it uses the enumerate* environment)
    \usepackage[normalem]{ulem} % ulem is needed to support strikethroughs (\sout)
                                % normalem makes italics be italics, not underlines
    

    
    
    % Colors for the hyperref package
    \definecolor{urlcolor}{rgb}{0,.145,.698}
    \definecolor{linkcolor}{rgb}{.71,0.21,0.01}
    \definecolor{citecolor}{rgb}{.12,.54,.11}

    % ANSI colors
    \definecolor{ansi-black}{HTML}{3E424D}
    \definecolor{ansi-black-intense}{HTML}{282C36}
    \definecolor{ansi-red}{HTML}{E75C58}
    \definecolor{ansi-red-intense}{HTML}{B22B31}
    \definecolor{ansi-green}{HTML}{00A250}
    \definecolor{ansi-green-intense}{HTML}{007427}
    \definecolor{ansi-yellow}{HTML}{DDB62B}
    \definecolor{ansi-yellow-intense}{HTML}{B27D12}
    \definecolor{ansi-blue}{HTML}{208FFB}
    \definecolor{ansi-blue-intense}{HTML}{0065CA}
    \definecolor{ansi-magenta}{HTML}{D160C4}
    \definecolor{ansi-magenta-intense}{HTML}{A03196}
    \definecolor{ansi-cyan}{HTML}{60C6C8}
    \definecolor{ansi-cyan-intense}{HTML}{258F8F}
    \definecolor{ansi-white}{HTML}{C5C1B4}
    \definecolor{ansi-white-intense}{HTML}{A1A6B2}

    % commands and environments needed by pandoc snippets
    % extracted from the output of `pandoc -s`
    \providecommand{\tightlist}{%
      \setlength{\itemsep}{0pt}\setlength{\parskip}{0pt}}
    \DefineVerbatimEnvironment{Highlighting}{Verbatim}{commandchars=\\\{\}}
    % Add ',fontsize=\small' for more characters per line
    \newenvironment{Shaded}{}{}
    \newcommand{\KeywordTok}[1]{\textcolor[rgb]{0.00,0.44,0.13}{\textbf{{#1}}}}
    \newcommand{\DataTypeTok}[1]{\textcolor[rgb]{0.56,0.13,0.00}{{#1}}}
    \newcommand{\DecValTok}[1]{\textcolor[rgb]{0.25,0.63,0.44}{{#1}}}
    \newcommand{\BaseNTok}[1]{\textcolor[rgb]{0.25,0.63,0.44}{{#1}}}
    \newcommand{\FloatTok}[1]{\textcolor[rgb]{0.25,0.63,0.44}{{#1}}}
    \newcommand{\CharTok}[1]{\textcolor[rgb]{0.25,0.44,0.63}{{#1}}}
    \newcommand{\StringTok}[1]{\textcolor[rgb]{0.25,0.44,0.63}{{#1}}}
    \newcommand{\CommentTok}[1]{\textcolor[rgb]{0.38,0.63,0.69}{\textit{{#1}}}}
    \newcommand{\OtherTok}[1]{\textcolor[rgb]{0.00,0.44,0.13}{{#1}}}
    \newcommand{\AlertTok}[1]{\textcolor[rgb]{1.00,0.00,0.00}{\textbf{{#1}}}}
    \newcommand{\FunctionTok}[1]{\textcolor[rgb]{0.02,0.16,0.49}{{#1}}}
    \newcommand{\RegionMarkerTok}[1]{{#1}}
    \newcommand{\ErrorTok}[1]{\textcolor[rgb]{1.00,0.00,0.00}{\textbf{{#1}}}}
    \newcommand{\NormalTok}[1]{{#1}}
    
    % Additional commands for more recent versions of Pandoc
    \newcommand{\ConstantTok}[1]{\textcolor[rgb]{0.53,0.00,0.00}{{#1}}}
    \newcommand{\SpecialCharTok}[1]{\textcolor[rgb]{0.25,0.44,0.63}{{#1}}}
    \newcommand{\VerbatimStringTok}[1]{\textcolor[rgb]{0.25,0.44,0.63}{{#1}}}
    \newcommand{\SpecialStringTok}[1]{\textcolor[rgb]{0.73,0.40,0.53}{{#1}}}
    \newcommand{\ImportTok}[1]{{#1}}
    \newcommand{\DocumentationTok}[1]{\textcolor[rgb]{0.73,0.13,0.13}{\textit{{#1}}}}
    \newcommand{\AnnotationTok}[1]{\textcolor[rgb]{0.38,0.63,0.69}{\textbf{\textit{{#1}}}}}
    \newcommand{\CommentVarTok}[1]{\textcolor[rgb]{0.38,0.63,0.69}{\textbf{\textit{{#1}}}}}
    \newcommand{\VariableTok}[1]{\textcolor[rgb]{0.10,0.09,0.49}{{#1}}}
    \newcommand{\ControlFlowTok}[1]{\textcolor[rgb]{0.00,0.44,0.13}{\textbf{{#1}}}}
    \newcommand{\OperatorTok}[1]{\textcolor[rgb]{0.40,0.40,0.40}{{#1}}}
    \newcommand{\BuiltInTok}[1]{{#1}}
    \newcommand{\ExtensionTok}[1]{{#1}}
    \newcommand{\PreprocessorTok}[1]{\textcolor[rgb]{0.74,0.48,0.00}{{#1}}}
    \newcommand{\AttributeTok}[1]{\textcolor[rgb]{0.49,0.56,0.16}{{#1}}}
    \newcommand{\InformationTok}[1]{\textcolor[rgb]{0.38,0.63,0.69}{\textbf{\textit{{#1}}}}}
    \newcommand{\WarningTok}[1]{\textcolor[rgb]{0.38,0.63,0.69}{\textbf{\textit{{#1}}}}}
    
    
    % Define a nice break command that doesn't care if a line doesn't already
    % exist.
    \def\br{\hspace*{\fill} \\* }
    % Math Jax compatability definitions
    \def\gt{>}
    \def\lt{<}
    % Document parameters
    \title{Education Is Significant Predictor of Median Wage, Martinez}
    
    
    

    % Pygments definitions
    
\makeatletter
\def\PY@reset{\let\PY@it=\relax \let\PY@bf=\relax%
    \let\PY@ul=\relax \let\PY@tc=\relax%
    \let\PY@bc=\relax \let\PY@ff=\relax}
\def\PY@tok#1{\csname PY@tok@#1\endcsname}
\def\PY@toks#1+{\ifx\relax#1\empty\else%
    \PY@tok{#1}\expandafter\PY@toks\fi}
\def\PY@do#1{\PY@bc{\PY@tc{\PY@ul{%
    \PY@it{\PY@bf{\PY@ff{#1}}}}}}}
\def\PY#1#2{\PY@reset\PY@toks#1+\relax+\PY@do{#2}}

\expandafter\def\csname PY@tok@w\endcsname{\def\PY@tc##1{\textcolor[rgb]{0.73,0.73,0.73}{##1}}}
\expandafter\def\csname PY@tok@c\endcsname{\let\PY@it=\textit\def\PY@tc##1{\textcolor[rgb]{0.25,0.50,0.50}{##1}}}
\expandafter\def\csname PY@tok@cp\endcsname{\def\PY@tc##1{\textcolor[rgb]{0.74,0.48,0.00}{##1}}}
\expandafter\def\csname PY@tok@k\endcsname{\let\PY@bf=\textbf\def\PY@tc##1{\textcolor[rgb]{0.00,0.50,0.00}{##1}}}
\expandafter\def\csname PY@tok@kp\endcsname{\def\PY@tc##1{\textcolor[rgb]{0.00,0.50,0.00}{##1}}}
\expandafter\def\csname PY@tok@kt\endcsname{\def\PY@tc##1{\textcolor[rgb]{0.69,0.00,0.25}{##1}}}
\expandafter\def\csname PY@tok@o\endcsname{\def\PY@tc##1{\textcolor[rgb]{0.40,0.40,0.40}{##1}}}
\expandafter\def\csname PY@tok@ow\endcsname{\let\PY@bf=\textbf\def\PY@tc##1{\textcolor[rgb]{0.67,0.13,1.00}{##1}}}
\expandafter\def\csname PY@tok@nb\endcsname{\def\PY@tc##1{\textcolor[rgb]{0.00,0.50,0.00}{##1}}}
\expandafter\def\csname PY@tok@nf\endcsname{\def\PY@tc##1{\textcolor[rgb]{0.00,0.00,1.00}{##1}}}
\expandafter\def\csname PY@tok@nc\endcsname{\let\PY@bf=\textbf\def\PY@tc##1{\textcolor[rgb]{0.00,0.00,1.00}{##1}}}
\expandafter\def\csname PY@tok@nn\endcsname{\let\PY@bf=\textbf\def\PY@tc##1{\textcolor[rgb]{0.00,0.00,1.00}{##1}}}
\expandafter\def\csname PY@tok@ne\endcsname{\let\PY@bf=\textbf\def\PY@tc##1{\textcolor[rgb]{0.82,0.25,0.23}{##1}}}
\expandafter\def\csname PY@tok@nv\endcsname{\def\PY@tc##1{\textcolor[rgb]{0.10,0.09,0.49}{##1}}}
\expandafter\def\csname PY@tok@no\endcsname{\def\PY@tc##1{\textcolor[rgb]{0.53,0.00,0.00}{##1}}}
\expandafter\def\csname PY@tok@nl\endcsname{\def\PY@tc##1{\textcolor[rgb]{0.63,0.63,0.00}{##1}}}
\expandafter\def\csname PY@tok@ni\endcsname{\let\PY@bf=\textbf\def\PY@tc##1{\textcolor[rgb]{0.60,0.60,0.60}{##1}}}
\expandafter\def\csname PY@tok@na\endcsname{\def\PY@tc##1{\textcolor[rgb]{0.49,0.56,0.16}{##1}}}
\expandafter\def\csname PY@tok@nt\endcsname{\let\PY@bf=\textbf\def\PY@tc##1{\textcolor[rgb]{0.00,0.50,0.00}{##1}}}
\expandafter\def\csname PY@tok@nd\endcsname{\def\PY@tc##1{\textcolor[rgb]{0.67,0.13,1.00}{##1}}}
\expandafter\def\csname PY@tok@s\endcsname{\def\PY@tc##1{\textcolor[rgb]{0.73,0.13,0.13}{##1}}}
\expandafter\def\csname PY@tok@sd\endcsname{\let\PY@it=\textit\def\PY@tc##1{\textcolor[rgb]{0.73,0.13,0.13}{##1}}}
\expandafter\def\csname PY@tok@si\endcsname{\let\PY@bf=\textbf\def\PY@tc##1{\textcolor[rgb]{0.73,0.40,0.53}{##1}}}
\expandafter\def\csname PY@tok@se\endcsname{\let\PY@bf=\textbf\def\PY@tc##1{\textcolor[rgb]{0.73,0.40,0.13}{##1}}}
\expandafter\def\csname PY@tok@sr\endcsname{\def\PY@tc##1{\textcolor[rgb]{0.73,0.40,0.53}{##1}}}
\expandafter\def\csname PY@tok@ss\endcsname{\def\PY@tc##1{\textcolor[rgb]{0.10,0.09,0.49}{##1}}}
\expandafter\def\csname PY@tok@sx\endcsname{\def\PY@tc##1{\textcolor[rgb]{0.00,0.50,0.00}{##1}}}
\expandafter\def\csname PY@tok@m\endcsname{\def\PY@tc##1{\textcolor[rgb]{0.40,0.40,0.40}{##1}}}
\expandafter\def\csname PY@tok@gh\endcsname{\let\PY@bf=\textbf\def\PY@tc##1{\textcolor[rgb]{0.00,0.00,0.50}{##1}}}
\expandafter\def\csname PY@tok@gu\endcsname{\let\PY@bf=\textbf\def\PY@tc##1{\textcolor[rgb]{0.50,0.00,0.50}{##1}}}
\expandafter\def\csname PY@tok@gd\endcsname{\def\PY@tc##1{\textcolor[rgb]{0.63,0.00,0.00}{##1}}}
\expandafter\def\csname PY@tok@gi\endcsname{\def\PY@tc##1{\textcolor[rgb]{0.00,0.63,0.00}{##1}}}
\expandafter\def\csname PY@tok@gr\endcsname{\def\PY@tc##1{\textcolor[rgb]{1.00,0.00,0.00}{##1}}}
\expandafter\def\csname PY@tok@ge\endcsname{\let\PY@it=\textit}
\expandafter\def\csname PY@tok@gs\endcsname{\let\PY@bf=\textbf}
\expandafter\def\csname PY@tok@gp\endcsname{\let\PY@bf=\textbf\def\PY@tc##1{\textcolor[rgb]{0.00,0.00,0.50}{##1}}}
\expandafter\def\csname PY@tok@go\endcsname{\def\PY@tc##1{\textcolor[rgb]{0.53,0.53,0.53}{##1}}}
\expandafter\def\csname PY@tok@gt\endcsname{\def\PY@tc##1{\textcolor[rgb]{0.00,0.27,0.87}{##1}}}
\expandafter\def\csname PY@tok@err\endcsname{\def\PY@bc##1{\setlength{\fboxsep}{0pt}\fcolorbox[rgb]{1.00,0.00,0.00}{1,1,1}{\strut ##1}}}
\expandafter\def\csname PY@tok@kc\endcsname{\let\PY@bf=\textbf\def\PY@tc##1{\textcolor[rgb]{0.00,0.50,0.00}{##1}}}
\expandafter\def\csname PY@tok@kd\endcsname{\let\PY@bf=\textbf\def\PY@tc##1{\textcolor[rgb]{0.00,0.50,0.00}{##1}}}
\expandafter\def\csname PY@tok@kn\endcsname{\let\PY@bf=\textbf\def\PY@tc##1{\textcolor[rgb]{0.00,0.50,0.00}{##1}}}
\expandafter\def\csname PY@tok@kr\endcsname{\let\PY@bf=\textbf\def\PY@tc##1{\textcolor[rgb]{0.00,0.50,0.00}{##1}}}
\expandafter\def\csname PY@tok@bp\endcsname{\def\PY@tc##1{\textcolor[rgb]{0.00,0.50,0.00}{##1}}}
\expandafter\def\csname PY@tok@fm\endcsname{\def\PY@tc##1{\textcolor[rgb]{0.00,0.00,1.00}{##1}}}
\expandafter\def\csname PY@tok@vc\endcsname{\def\PY@tc##1{\textcolor[rgb]{0.10,0.09,0.49}{##1}}}
\expandafter\def\csname PY@tok@vg\endcsname{\def\PY@tc##1{\textcolor[rgb]{0.10,0.09,0.49}{##1}}}
\expandafter\def\csname PY@tok@vi\endcsname{\def\PY@tc##1{\textcolor[rgb]{0.10,0.09,0.49}{##1}}}
\expandafter\def\csname PY@tok@vm\endcsname{\def\PY@tc##1{\textcolor[rgb]{0.10,0.09,0.49}{##1}}}
\expandafter\def\csname PY@tok@sa\endcsname{\def\PY@tc##1{\textcolor[rgb]{0.73,0.13,0.13}{##1}}}
\expandafter\def\csname PY@tok@sb\endcsname{\def\PY@tc##1{\textcolor[rgb]{0.73,0.13,0.13}{##1}}}
\expandafter\def\csname PY@tok@sc\endcsname{\def\PY@tc##1{\textcolor[rgb]{0.73,0.13,0.13}{##1}}}
\expandafter\def\csname PY@tok@dl\endcsname{\def\PY@tc##1{\textcolor[rgb]{0.73,0.13,0.13}{##1}}}
\expandafter\def\csname PY@tok@s2\endcsname{\def\PY@tc##1{\textcolor[rgb]{0.73,0.13,0.13}{##1}}}
\expandafter\def\csname PY@tok@sh\endcsname{\def\PY@tc##1{\textcolor[rgb]{0.73,0.13,0.13}{##1}}}
\expandafter\def\csname PY@tok@s1\endcsname{\def\PY@tc##1{\textcolor[rgb]{0.73,0.13,0.13}{##1}}}
\expandafter\def\csname PY@tok@mb\endcsname{\def\PY@tc##1{\textcolor[rgb]{0.40,0.40,0.40}{##1}}}
\expandafter\def\csname PY@tok@mf\endcsname{\def\PY@tc##1{\textcolor[rgb]{0.40,0.40,0.40}{##1}}}
\expandafter\def\csname PY@tok@mh\endcsname{\def\PY@tc##1{\textcolor[rgb]{0.40,0.40,0.40}{##1}}}
\expandafter\def\csname PY@tok@mi\endcsname{\def\PY@tc##1{\textcolor[rgb]{0.40,0.40,0.40}{##1}}}
\expandafter\def\csname PY@tok@il\endcsname{\def\PY@tc##1{\textcolor[rgb]{0.40,0.40,0.40}{##1}}}
\expandafter\def\csname PY@tok@mo\endcsname{\def\PY@tc##1{\textcolor[rgb]{0.40,0.40,0.40}{##1}}}
\expandafter\def\csname PY@tok@ch\endcsname{\let\PY@it=\textit\def\PY@tc##1{\textcolor[rgb]{0.25,0.50,0.50}{##1}}}
\expandafter\def\csname PY@tok@cm\endcsname{\let\PY@it=\textit\def\PY@tc##1{\textcolor[rgb]{0.25,0.50,0.50}{##1}}}
\expandafter\def\csname PY@tok@cpf\endcsname{\let\PY@it=\textit\def\PY@tc##1{\textcolor[rgb]{0.25,0.50,0.50}{##1}}}
\expandafter\def\csname PY@tok@c1\endcsname{\let\PY@it=\textit\def\PY@tc##1{\textcolor[rgb]{0.25,0.50,0.50}{##1}}}
\expandafter\def\csname PY@tok@cs\endcsname{\let\PY@it=\textit\def\PY@tc##1{\textcolor[rgb]{0.25,0.50,0.50}{##1}}}

\def\PYZbs{\char`\\}
\def\PYZus{\char`\_}
\def\PYZob{\char`\{}
\def\PYZcb{\char`\}}
\def\PYZca{\char`\^}
\def\PYZam{\char`\&}
\def\PYZlt{\char`\<}
\def\PYZgt{\char`\>}
\def\PYZsh{\char`\#}
\def\PYZpc{\char`\%}
\def\PYZdl{\char`\$}
\def\PYZhy{\char`\-}
\def\PYZsq{\char`\'}
\def\PYZdq{\char`\"}
\def\PYZti{\char`\~}
% for compatibility with earlier versions
\def\PYZat{@}
\def\PYZlb{[}
\def\PYZrb{]}
\makeatother


    % Exact colors from NB
    \definecolor{incolor}{rgb}{0.0, 0.0, 0.5}
    \definecolor{outcolor}{rgb}{0.545, 0.0, 0.0}



    
    % Prevent overflowing lines due to hard-to-break entities
    \sloppy 
    % Setup hyperref package
    \hypersetup{
      breaklinks=true,  % so long urls are correctly broken across lines
      colorlinks=true,
      urlcolor=urlcolor,
      linkcolor=linkcolor,
      citecolor=citecolor,
      }
    % Slightly bigger margins than the latex defaults
    
    \geometry{verbose,tmargin=1in,bmargin=1in,lmargin=1in,rmargin=1in}
    
    

    \begin{document}
    
    
    \maketitle
    
    

    
    \section{Higher Education Means Higher Income
Overall}\label{higher-education-means-higher-income-overall}

\subsubsection{Male Wage Earners in Atlanta 2003 -
2009}\label{male-wage-earners-in-atlanta-2003---2009}

By: Paulo Martinez

Prepared for ERPi, 12-2-2018

\subsection{Objective: Discuss the relationship between education and
wages}\label{objective-discuss-the-relationship-between-education-and-wages}

Based on a dataset of 3,000 observations describing men's incomes in
Atlanta.

    \subsubsection{First Thing's First: How Trustworthy is this
data?}\label{first-things-first-how-trustworthy-is-this-data}

\subsubsection{It's unclear}\label{its-unclear}

DATA = dspe\_wage.csv It contains the following 11 variables: - year:
Year that wage information was recorded - age: Age of worker - maritl: A
factor with levels 1. Never Married 2. Married 3. Widowed 4. Divorced
and 5. Separated indicating marital status - race: A factor with levels
1. White 2. Black 3. Asian and 4. Other indicating race - education: A
factor with levels 1. \textless{} HS Grad 2. HS Grad 3. Some College 4.
College Grad and 5. Advanced Degree indicating education level - region:
Region of the country (mid-atlantic only) - jobclass: A factor with
levels 1. Industrial and 2. Information indicating type of job - health:
A factor with levels 1. \textless{}=Good and 2. \textgreater{}=Very Good
indicating health level of worker - health\_ins: A factor with levels 1.
Yes and 2. No indicating whether worker has health insurance - logwage:
Log of workers wage - wage: Workers raw wage - \emph{'Unnamed: 0' - Most
likely is the record identifier from the parent database. (see EDA
notebook in parent directory)'}

    \section{Finding 1:}\label{finding-1}

\subsection{⚠️ The "timestamps" for 2003 \& 2005 May have been mixed up.
And we need to find out
why.}\label{the-timestamps-for-2003-2005-may-have-been-mixed-up.-and-we-need-to-find-out-why.}

\subsection{Recommendation 1: Double check the process by which this
data came to
be.}\label{recommendation-1-double-check-the-process-by-which-this-data-came-to-be.}

\subsubsection{The Data came with an unexpected variable 'Unnamed:
0'.}\label{the-data-came-with-an-unexpected-variable-unnamed-0.}

\begin{itemize}
\item ~
  \subparagraph{I suspect it is just the record numbers from the
  database this file was pulled
  from.}\label{i-suspect-it-is-just-the-record-numbers-from-the-database-this-file-was-pulled-from.}
\item ~
  \paragraph{⚠️ Nonethless it was unnanounced and it remains unclear if
  this was an (un)intentional data
  leak.}\label{nonethless-it-was-unnanounced-and-it-remains-unclear-if-this-was-an-unintentional-data-leak.}
\item ~
  \paragraph{These record identifiers also are the only clue that the
  year tags from 2003 may have been mixed up with those from
  2005}\label{these-record-identifiers-also-are-the-only-clue-that-the-year-tags-from-2003-may-have-been-mixed-up-with-those-from-2005}
\end{itemize}

    \begin{Verbatim}[commandchars=\\\{\}]
{\color{incolor}In [{\color{incolor}55}]:} \PY{k+kn}{import} \PY{n+nn}{pandas} \PY{k}{as} \PY{n+nn}{pd}
         \PY{k+kn}{import} \PY{n+nn}{numpy} \PY{k}{as} \PY{n+nn}{np}
         \PY{k+kn}{import} \PY{n+nn}{matplotlib}\PY{n+nn}{.}\PY{n+nn}{pyplot} \PY{k}{as} \PY{n+nn}{plt}
         \PY{k+kn}{import} \PY{n+nn}{seaborn} \PY{k}{as} \PY{n+nn}{sns}
         \PY{n}{wages\PYZus{}df} \PY{o}{=} \PY{n}{pd}\PY{o}{.}\PY{n}{read\PYZus{}csv}\PY{p}{(}\PY{l+s+s1}{\PYZsq{}}\PY{l+s+s1}{data/dspe\PYZus{}wage.csv}\PY{l+s+s1}{\PYZsq{}}\PY{p}{)}
\end{Verbatim}


    \textbf{Notice} that the record numbers from the parent database are
organized in descending order from 2003 - 2005, but in ascending order
from 2006 - 2009. This might be explained because of a bug in the way
the data queried.

\textbf{This also makes it unwise to rely on chronological analyses
until the matter is resolved and presents a constraint on the analysis
possible}

    \begin{Verbatim}[commandchars=\\\{\}]
{\color{incolor}In [{\color{incolor}56}]:} \PY{n}{sns}\PY{o}{.}\PY{n}{lmplot}\PY{p}{(}\PY{n}{x} \PY{o}{=} \PY{l+s+s1}{\PYZsq{}}\PY{l+s+s1}{year}\PY{l+s+s1}{\PYZsq{}}\PY{p}{,} \PY{n}{y} \PY{o}{=} \PY{l+s+s1}{\PYZsq{}}\PY{l+s+s1}{Unnamed: 0}\PY{l+s+s1}{\PYZsq{}}\PY{p}{,} \PY{n}{data} \PY{o}{=} \PY{n}{wages\PYZus{}df}\PY{p}{)}
         \PY{n}{plt}\PY{o}{.}\PY{n}{title}\PY{p}{(}\PY{l+s+s1}{\PYZsq{}}\PY{l+s+s1}{Scatter plot of year vs database record id}\PY{l+s+s1}{\PYZsq{}}\PY{p}{)}
         \PY{n}{plt}\PY{o}{.}\PY{n}{show}\PY{p}{(}\PY{p}{)}
\end{Verbatim}


    \begin{center}
    \adjustimage{max size={0.9\linewidth}{0.9\paperheight}}{output_5_0.png}
    \end{center}
    { \hspace*{\fill} \\}
    
    The above plot is a scatter plot with an ordinary least square
regression linear model showing the anticipated ascending order of
record ids.

    \section{Finding 2:}\label{finding-2}

\begin{itemize}
\item ~
  \subsection{Workers who graduated highschool but never went to college
  are the largest group in this
  data}\label{workers-who-graduated-highschool-but-never-went-to-college-are-the-largest-group-in-this-data}
\item ~
  \subsubsection{Workers who graduated college are the second largest
  group}\label{workers-who-graduated-college-are-the-second-largest-group}
\item ~
  \paragraph{They are followed closely by workers who went to but have
  not comopleted
  college}\label{they-are-followed-closely-by-workers-who-went-to-but-have-not-comopleted-college}
\item
  \textbf{The next largest group are workers with advanced degrees}
\item
  And the smallest group are workers who did not graduate highschool
\end{itemize}

    \begin{Verbatim}[commandchars=\\\{\}]
{\color{incolor}In [{\color{incolor}57}]:} \PY{n}{sns}\PY{o}{.}\PY{n}{set\PYZus{}style}\PY{p}{(}\PY{l+s+s1}{\PYZsq{}}\PY{l+s+s1}{darkgrid}\PY{l+s+s1}{\PYZsq{}}\PY{p}{)}
         \PY{n}{f}\PY{p}{,} \PY{n}{ax} \PY{o}{=} \PY{n}{plt}\PY{o}{.}\PY{n}{subplots}\PY{p}{(}\PY{n}{figsize} \PY{o}{=} \PY{p}{(}\PY{l+m+mi}{20}\PY{p}{,} \PY{l+m+mi}{5}\PY{p}{)}\PY{p}{)}
         \PY{n}{sns}\PY{o}{.}\PY{n}{countplot}\PY{p}{(}\PY{n}{x} \PY{o}{=} \PY{l+s+s1}{\PYZsq{}}\PY{l+s+s1}{education}\PY{l+s+s1}{\PYZsq{}}\PY{p}{,} 
                       \PY{n}{order} \PY{o}{=} \PY{n+nb}{sorted}\PY{p}{(}\PY{n+nb}{set}\PY{p}{(}\PY{n}{wages\PYZus{}df}\PY{o}{.}\PY{n}{education}\PY{p}{)}\PY{p}{)}\PY{p}{,}
                       \PY{n}{data} \PY{o}{=} \PY{n}{wages\PYZus{}df}\PY{p}{,} 
                       \PY{n}{palette} \PY{o}{=} \PY{l+s+s1}{\PYZsq{}}\PY{l+s+s1}{viridis}\PY{l+s+s1}{\PYZsq{}}\PY{p}{,}
                       \PY{n}{alpha} \PY{o}{=} \PY{o}{.}\PY{l+m+mi}{7}
                      \PY{p}{)}
         \PY{n}{plt}\PY{o}{.}\PY{n}{title}\PY{p}{(}\PY{l+s+s1}{\PYZsq{}}\PY{l+s+s1}{Education Count Plot 03 \PYZhy{} 09}\PY{l+s+s1}{\PYZsq{}}\PY{p}{)}
         \PY{n}{plt}\PY{o}{.}\PY{n}{show}\PY{p}{(}\PY{p}{)}
\end{Verbatim}


    \begin{center}
    \adjustimage{max size={0.9\linewidth}{0.9\paperheight}}{output_8_0.png}
    \end{center}
    { \hspace*{\fill} \\}
    
    \section{Finding 3:}\label{finding-3}

\subsection{The median wage overall is 105 and the best way to summarize
wages}\label{the-median-wage-overall-is-105-and-the-best-way-to-summarize-wages}

\begin{itemize}
\tightlist
\item
  \textbf{There distributuion is skewed to the right, meaning the mean
  wage will be biased to be a bit higher}
\item
  The data reflects a 95\% confidence interval for the grand median
  between:
\end{itemize}

{[}104.46323322, 106.92181409{]}

    \begin{Verbatim}[commandchars=\\\{\}]
{\color{incolor}In [{\color{incolor}58}]:} \PY{n}{f}\PY{p}{,} \PY{n}{ax} \PY{o}{=} \PY{n}{plt}\PY{o}{.}\PY{n}{subplots}\PY{p}{(}\PY{n}{figsize} \PY{o}{=} \PY{p}{(}\PY{l+m+mi}{20}\PY{p}{,} \PY{l+m+mi}{5}\PY{p}{)}\PY{p}{)}
         \PY{n}{wages\PYZus{}df}\PY{o}{.}\PY{n}{wage}\PY{o}{.}\PY{n}{hist}\PY{p}{(}\PY{n}{color} \PY{o}{=} \PY{l+s+s1}{\PYZsq{}}\PY{l+s+s1}{green}\PY{l+s+s1}{\PYZsq{}}\PY{p}{,} \PY{n}{alpha} \PY{o}{=} \PY{o}{.}\PY{l+m+mi}{7}\PY{p}{)}
         \PY{n}{plt}\PY{o}{.}\PY{n}{title}\PY{p}{(}\PY{l+s+s1}{\PYZsq{}}\PY{l+s+s1}{Histogram of wage distribution}\PY{l+s+s1}{\PYZsq{}}\PY{p}{)}
         \PY{n}{plt}\PY{o}{.}\PY{n}{show}\PY{p}{(}\PY{p}{)}
\end{Verbatim}


    \begin{center}
    \adjustimage{max size={0.9\linewidth}{0.9\paperheight}}{output_10_0.png}
    \end{center}
    { \hspace*{\fill} \\}
    
    \begin{Verbatim}[commandchars=\\\{\}]
{\color{incolor}In [{\color{incolor}59}]:} \PY{n+nb}{print}\PY{p}{(}\PY{n}{wages\PYZus{}df}\PY{o}{.}\PY{n}{wage}\PY{o}{.}\PY{n}{describe}\PY{p}{(}\PY{p}{)}\PY{p}{)}
\end{Verbatim}


    \begin{Verbatim}[commandchars=\\\{\}]
count    3000.000000
mean      111.703608
std        41.728595
min        20.085537
25\%        85.383940
50\%       104.921507
75\%       128.680488
max       318.342430
Name: wage, dtype: float64

    \end{Verbatim}

    \section{Finding 4:}\label{finding-4}

\begin{itemize}
\item ~
  \subsection{An advanced degree improves your odds, but is no guarantee
  of earning more than 100 or even
  50}\label{an-advanced-degree-improves-your-odds-but-is-no-guarantee-of-earning-more-than-100-or-even-50}
\item ~
  \subsection{But the absence of a highschool diploma is an invisible
  wall keeping wages under 150; most of them under
  100.}\label{but-the-absence-of-a-highschool-diploma-is-an-invisible-wall-keeping-wages-under-150-most-of-them-under-100.}
\end{itemize}

    \begin{Verbatim}[commandchars=\\\{\}]
{\color{incolor}In [{\color{incolor}63}]:} \PY{n}{f}\PY{p}{,} \PY{n}{ax} \PY{o}{=} \PY{n}{plt}\PY{o}{.}\PY{n}{subplots}\PY{p}{(}\PY{n}{figsize} \PY{o}{=} \PY{p}{(}\PY{l+m+mi}{20}\PY{p}{,} \PY{l+m+mi}{10}\PY{p}{)}\PY{p}{)}
         \PY{c+c1}{\PYZsh{}sns.swarmplot(x = \PYZsq{}education\PYZsq{}, palette = \PYZsq{}viridis\PYZsq{},y = \PYZsq{}wage\PYZsq{}, data = wages\PYZus{}df, order = sorted(set(wages\PYZus{}df.education)),edgecolor = \PYZsq{}white\PYZsq{},linewidth = .7,)}
         \PY{n}{sns}\PY{o}{.}\PY{n}{boxplot}\PY{p}{(}\PY{n}{x} \PY{o}{=} \PY{l+s+s1}{\PYZsq{}}\PY{l+s+s1}{education}\PY{l+s+s1}{\PYZsq{}}\PY{p}{,} \PY{n}{palette} \PY{o}{=} \PY{l+s+s1}{\PYZsq{}}\PY{l+s+s1}{viridis}\PY{l+s+s1}{\PYZsq{}}\PY{p}{,}\PY{n}{y} \PY{o}{=} \PY{l+s+s1}{\PYZsq{}}\PY{l+s+s1}{wage}\PY{l+s+s1}{\PYZsq{}}\PY{p}{,} \PY{n}{data} \PY{o}{=} \PY{n}{wages\PYZus{}df}\PY{p}{,} \PY{n}{order} \PY{o}{=} \PY{n+nb}{sorted}\PY{p}{(}\PY{n+nb}{set}\PY{p}{(}\PY{n}{wages\PYZus{}df}\PY{o}{.}\PY{n}{education}\PY{p}{)}\PY{p}{)}\PY{p}{,}\PY{n}{saturation} \PY{o}{=} \PY{o}{.}\PY{l+m+mi}{3}\PY{p}{,}\PY{p}{)}
         \PY{n}{sns}\PY{o}{.}\PY{n}{pointplot}\PY{p}{(}\PY{n}{x} \PY{o}{=} \PY{l+s+s1}{\PYZsq{}}\PY{l+s+s1}{education}\PY{l+s+s1}{\PYZsq{}}\PY{p}{,} \PY{n}{y} \PY{o}{=} \PY{l+s+s1}{\PYZsq{}}\PY{l+s+s1}{wage}\PY{l+s+s1}{\PYZsq{}}\PY{p}{,} \PY{n}{data} \PY{o}{=} \PY{n}{wages\PYZus{}df}\PY{p}{,} \PY{n}{color} \PY{o}{=} \PY{l+s+s1}{\PYZsq{}}\PY{l+s+s1}{yellow}\PY{l+s+s1}{\PYZsq{}}\PY{p}{,} \PY{n}{estimator} \PY{o}{=} \PY{n}{np}\PY{o}{.}\PY{n}{median}\PY{p}{,} \PY{n}{order} \PY{o}{=} \PY{n+nb}{sorted}\PY{p}{(}\PY{n+nb}{set}\PY{p}{(}\PY{n}{wages\PYZus{}df}\PY{o}{.}\PY{n}{education}\PY{p}{)}\PY{p}{)}\PY{p}{)}
         \PY{n}{plt}\PY{o}{.}\PY{n}{title}\PY{p}{(}\PY{l+s+s1}{\PYZsq{}}\PY{l+s+s1}{Box Plot Distribution of Wages Grouped by Education}\PY{l+s+s1}{\PYZsq{}}\PY{p}{)}
         \PY{n}{plt}\PY{o}{.}\PY{n}{legend}\PY{p}{(}\PY{p}{)}
         \PY{n}{plt}\PY{o}{.}\PY{n}{show}\PY{p}{(}\PY{p}{)}
\end{Verbatim}


    \begin{Verbatim}[commandchars=\\\{\}]
No handles with labels found to put in legend.

    \end{Verbatim}

    \begin{center}
    \adjustimage{max size={0.9\linewidth}{0.9\paperheight}}{output_13_1.png}
    \end{center}
    { \hspace*{\fill} \\}
    
    \textbf{What this plot shows:} Each "Box and whiskers" plot shows the
summary statistics for the wage distribution. - The horizontal line in
the middle of the box represents the mean of the distribution - the
filled-in rectangels below and above the mean are the 25\% - 75\%
inter-quartile range - and the horizontal lines above and below that
represent the bottom and top quartiles - the diamonds above and below
the quartiles represent outliers in that distribution.

    \subsubsection{The median income for '1. \textless{} HS Grad' is:
81}\label{the-median-income-for-1.-hs-grad-is-81}

With a 95\% confidence interval of {[}79.12368356 82.6796373 {]}

    \subsubsection{The median income for '2. HS Grad' is about 13 higher at:
94}\label{the-median-income-for-2.-hs-grad-is-about-13-higher-at-94}

With a 95\% confidence interval of {[}91.69922611 94.07271475{]}

    \subsubsection{The median income for '3. Some College' is about 11
higher at:
105}\label{the-median-income-for-3.-some-college-is-about-11-higher-at-105}

With a 95 \% confidence interval of {[}101.82435208 106.92181409{]}

    \subsubsection{The median income for '4 College Grad' is about 14 higher
at:
119}\label{the-median-income-for-4-college-grad-is-about-14-higher-at-119}

With a 95\% confidence interval of {[}118.01975334 123.08969985{]}

    \subsubsection{The median income for '5. Advanced Degree' is 21 higher
at:
142}\label{the-median-income-for-5.-advanced-degree-is-21-higher-at-142}

With a 95\% confidence interval of {[}134.70537512 145.80321985{]}

    \section{And these differences in median income for each educational
population have been validated using Mood's median test (a special
application of the chi-squared
statistic).}\label{and-these-differences-in-median-income-for-each-educational-population-have-been-validated-using-moods-median-test-a-special-application-of-the-chi-squared-statistic.}

\subsection{The odds of the education not being a predictor of median
wage but still looking the way it does in this data as a fluke are
astronomically low:
7.530594260145932e-116}\label{the-odds-of-the-education-not-being-a-predictor-of-median-wage-but-still-looking-the-way-it-does-in-this-data-as-a-fluke-are-astronomically-low-7.530594260145932e-116}

\begin{itemize}
\tightlist
\item
  However more analysis is required to determine what this means for the
  confidence interval.

  \begin{itemize}
  \tightlist
  \item
    Since the wage distributions were decidedly not normal. I had to use
    a different test of the significance of each median value than the
    more common z-tests or t-tests.
  \end{itemize}
\end{itemize}

    \begin{Verbatim}[commandchars=\\\{\}]
{\color{incolor}In [{\color{incolor}54}]:} \PY{c+c1}{\PYZsh{} list comprehension to generate and populate new column}
         \PY{n}{wages\PYZus{}df}\PY{p}{[}\PY{l+s+s1}{\PYZsq{}}\PY{l+s+s1}{ed\PYZus{}numeric}\PY{l+s+s1}{\PYZsq{}}\PY{p}{]} \PY{o}{=} \PY{p}{[}\PY{n+nb}{int}\PY{p}{(}\PY{n}{v}\PY{p}{[}\PY{l+m+mi}{0}\PY{p}{]}\PY{p}{)} \PY{k}{for} \PY{n}{v} \PY{o+ow}{in} \PY{n}{wages\PYZus{}df}\PY{o}{.}\PY{n}{education}\PY{p}{]}
         
         \PY{c+c1}{\PYZsh{} the stat used is Pearson\PYZsq{}s chi\PYZhy{}squared statistic.}
         \PY{k+kn}{from} \PY{n+nn}{scipy}\PY{n+nn}{.}\PY{n+nn}{stats} \PY{k}{import} \PY{n}{median\PYZus{}test}
         \PY{n}{stat}\PY{p}{,} \PY{n}{p}\PY{p}{,} \PY{n}{med}\PY{p}{,} \PY{n}{tbl} \PY{o}{=} \PY{n}{median\PYZus{}test}\PY{p}{(}
             \PY{n}{wages\PYZus{}df}\PY{o}{.}\PY{n}{wage}\PY{p}{[}\PY{n}{wages\PYZus{}df}\PY{o}{.}\PY{n}{ed\PYZus{}numeric} \PY{o}{==} \PY{l+m+mi}{1}\PY{p}{]}\PY{p}{,} 
             \PY{n}{wages\PYZus{}df}\PY{o}{.}\PY{n}{wage}\PY{p}{[}\PY{n}{wages\PYZus{}df}\PY{o}{.}\PY{n}{ed\PYZus{}numeric} \PY{o}{==} \PY{l+m+mi}{2}\PY{p}{]}\PY{p}{,}
             \PY{n}{wages\PYZus{}df}\PY{o}{.}\PY{n}{wage}\PY{p}{[}\PY{n}{wages\PYZus{}df}\PY{o}{.}\PY{n}{ed\PYZus{}numeric} \PY{o}{==} \PY{l+m+mi}{3}\PY{p}{]}\PY{p}{,}
             \PY{n}{wages\PYZus{}df}\PY{o}{.}\PY{n}{wage}\PY{p}{[}\PY{n}{wages\PYZus{}df}\PY{o}{.}\PY{n}{ed\PYZus{}numeric} \PY{o}{==} \PY{l+m+mi}{4}\PY{p}{]}\PY{p}{,}
             \PY{n}{wages\PYZus{}df}\PY{o}{.}\PY{n}{wage}\PY{p}{[}\PY{n}{wages\PYZus{}df}\PY{o}{.}\PY{n}{ed\PYZus{}numeric} \PY{o}{==} \PY{l+m+mi}{5}\PY{p}{]}\PY{p}{,}
         \PY{p}{)}
         \PY{n+nb}{print}\PY{p}{(}\PY{l+s+s1}{\PYZsq{}}\PY{l+s+s1}{grand med:}\PY{l+s+s1}{\PYZsq{}}\PY{p}{,} \PY{n}{med}\PY{p}{)}
         \PY{n+nb}{print}\PY{p}{(}\PY{l+s+s1}{\PYZsq{}}\PY{l+s+s1}{\PYZsq{}}\PY{p}{)}
         \PY{n+nb}{print}\PY{p}{(}\PY{l+s+s1}{\PYZsq{}}\PY{l+s+s1}{sample\PYZhy{}values}\PY{l+s+se}{\PYZbs{}n}\PY{l+s+s1}{\PYZsq{}}\PY{p}{,} \PY{l+s+s1}{\PYZsq{}}\PY{l+s+s1}{above, below}\PY{l+s+s1}{\PYZsq{}}\PY{p}{)}
         \PY{n+nb}{print}\PY{p}{(}\PY{n}{tbl}\PY{p}{)}
         \PY{n+nb}{print}\PY{p}{(}\PY{l+s+s1}{\PYZsq{}}\PY{l+s+s1}{\PYZsq{}}\PY{p}{)}
         \PY{n+nb}{print}\PY{p}{(}\PY{n}{stat}\PY{p}{)}
         \PY{n+nb}{print}\PY{p}{(}\PY{l+s+s1}{\PYZsq{}}\PY{l+s+s1}{p: }\PY{l+s+s1}{\PYZsq{}}\PY{p}{,} \PY{n}{p}\PY{p}{)}
         \PY{n+nb}{print}\PY{p}{(}\PY{l+s+s1}{\PYZsq{}}\PY{l+s+s1}{p\PYZhy{}value: the probability of getting our stat when all the arrays actually have the same median}\PY{l+s+s1}{\PYZsq{}}\PY{p}{)}
\end{Verbatim}


    \begin{Verbatim}[commandchars=\\\{\}]
grand med: 104.921506533664

sample-values
 above, below
[[ 46 302 312 464 359]
 [222 669 338 221  67]]

541.3710842640019
p:  7.530594260145932e-116
p-value: the probability of getting our stat when all the arrays actually have the same median

    \end{Verbatim}

    For a more detailed look at the work and methodology producing these
results please consult the other notebooks in this repo:


    % Add a bibliography block to the postdoc
    
    
    
    \end{document}
